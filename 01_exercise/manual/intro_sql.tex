\documentclass[12pt, letterpaper]{article}
\usepackage[utf8]{inputenc}
\usepackage{hyperref}
\usepackage{enumerate}
\usepackage{listings}
\usepackage{fancyvrb}
\usepackage[T1]{fontenc}
\usepackage[document]{ragged2e} % keep all left

\usepackage{pmboxdraw} %verbatim utf8

% add cache=false]
% if issues:
\usepackage[cache=false]{minted} % yaml syntax highlighting

%\usepackage{tikz}
%\usetikzlibrary{matrix,positioning}

\newenvironment{markdown}%
    {\VerbatimEnvironment\begin{VerbatimOut}{tmp.markdown}}%
    {\end{VerbatimOut}%
        \immediate\write18{pandoc tmp.markdown -t latex -o tmp.tex}%
        \input{tmp.tex}}

\providecommand{\tightlist}{%
  \setlength{\itemsep}{0pt}\setlength{\parskip}{0pt}}

\newcommand*{\email}[1]{\href{mailto:#1}{\nolinkurl{#1}} }

\title{{\Huge \textbf{{SQL} 1}}}
%\author{Wojciech Barczynski}
\date{}


\begin{document}

\begin{titlepage}
\maketitle
\end{titlepage}

\tableofcontents
\pagebreak
\section{Run Mysql database}

The first step is to learn how to run a mysql database on your workstation. To not spend too much time on the configuration, we will use docker to run your mysql instance.

\subsection{Verify whether your laptop setup}

Let's check whether we have correctly configured Docker, open your terminal and run the following command:

\begin{minted}{bash}
# on macOS you do not use sudo to run docker
# commands
sudo docker ps

# you should see:
CONTAINER ID   IMAGE     COMMAND   CREATED   STATUS    PORTS     NAMES

\end{minted}

\subsection{Run your instance}
To start your mysql instance with Docker, run the following command:

\begin{minted}{bash}
sudo docker run --name wsb-mysql \
   -e MYSQL_ROOT_PASSWORD=nomoresecret \
   -p 3306:3306 \
   -d mysql:8

# check whether you see your database running
sudo docker ps
\end{minted}

Notice:
\begin{itemize}
\item to stop the Mysql database in Docker: \mintinline{bash}{docker stop wsb-mysql}
\item to start your Mysql database in Docker: \mintinline{bash}{docker start wsb-mysql}.
\end{itemize}

\subsection{Mysql {\small CLI}}

You need to install the command line interface mysql package first:

\begin{minted}{bash}
sudo apt-get update
sudo apt install -y mysql-client
\end{minted}

To open the connection to your database:

\begin{minted}{bash}
# your password is: nomoresecret
mysql -u root -h 127.0.0.1 -p
\end{minted}

In the mysql console, please execute the following command: \mintinline{sql}{show databases;}:

\begin{minted}{text}
mysql> show databases;

+--------------------+
| Database           |
+--------------------+
| information_schema |
| mysql              |
| performance_schema |
| sys                |
+--------------------+
4 rows in set (0.00 sec)
\end{minted}

\subsection{Mysql Workbench}

We can also use a grafical interface to work with Mysql. There are many available tools, today we will use mysql-workbench\footnote{You might use \href{https://sequelpro.com/download}{sequelpro.com} as well}:

\begin{minted}{bash}
sudo apt update
sudo snap install mysql-workbench-community
snap connect mysql-workbench-community:password-manager-service
\end{minted}

You can run the workbench from your terminal:

\begin{minted}{bash}
# if we cannot see fonts:
export LANG=en_US
mysql-workbench-community &
\end{minted}

In the graphical interface select: \emph{Local Instance 3306} (user \emph{root}). To verify that everyhing works, please run the following commands:

\begin{minted}{sql}
-- 1
show databases;
-- 2
use mysql;
-- 3
show tables;
-- 4
select * from user;
\end{minted}

% \begin{itemize}
% \item \mintinline{sql}{}
% \item \mintinline{sql}{}
% \item \mintinline{sql}{}
% \item \mintinline{sql}{}
% \end{itemize}

Note down what users we have in the system.
\bigskip
\bigskip
\bigskip

\textbf{Notice}: You should not work as a database user \textbf{root}, you should always create a dedicated admin user for yourself.

%%%%%%%%%%%%%%%%%%%%%%
\section{Recap}

\subsection{Data Types}

\begin{itemize}
\item Characters: CHAR(20), VARCHAR(50), TEXT
\item Numbers: INT, BIGINT, SMALLINT, FLOAT
\item Logical: BOOL
\item Others: MONEY, DATETIME...
\end{itemize}

\subsection{Tables, views, and relations}

\begin{itemize}
\item relation or table (defined by a schema)
\item rows
\item each column (has a type)
\item key keys
\end{itemize}

%%%%%%%%%%%%%%%%%%%%%%
\section{Create your database}

Let's build our own database.

\bigskip
1. Open the workbench and create your database with a prefix: \verb|wsb|.

\begin{minted}{sql}
-- 1: do not forget to change the name
create database wsbnatalia

-- 2:
show databases;

-- 3:
use wsbnatalia
\end{minted}

\bigskip
2. Create your first table (in the schemas, choose your database as the default target for your commands):

\begin{minted}{sql}
CREATE TABLE Products (
  ProductID CHAR(20),
  ProductName VARCHAR(50),
  Price  float,
  Category VARCHAR(50),
  SuplierName VARCHAR(50),
  PRIMARY KEY (ProductID))
\end{minted}

\bigskip

3. Let's add one product:

\begin{minted}{sql}
INSERT INTO Products(ProductID, ProductName, Price,
    Category, SuplierName) VALUES (
    '123',
    'Milk',
    5.9,
    'Dairy Products',
    'Mlekovita'
)
\end{minted}

\bigskip

3. Let's add one product:

\begin{minted}{sql}
select * from Product;
\end{minted}

\bigskip
4. Please create new table - \verb|Suppliers| and add two rows with the following attributes:

\begin{itemize}
\item SupplierID (key)
\item SupplierName
\item ContactName
\item Country
\end{itemize}

%%%%%%%%%%%%%%%%%%%%%%
\section{Ecommerce database}

To work with a larger database, we will use the database behind w3schools SQL tutorial (\href{https://www.w3schools.com/sql/default.asp}{w3schools.com/sql}).

\bigskip

1. Download \emph{w3schools.sql} z \emph{https://github.com/wojciech11/w3schools-databasel} (open the file in browser and choose \verb|Raw|):

\begin{minted}{bash}
# long URL
wget https://raw.githubuse..
ls

w3schools.sql
\end{minted}

\bigskip
2. Let's load the database through \small{CLI}:

\begin{minted}{bash}
# your password is: nomoresecret
$  mysql -u root -h 127.0.0.1 -p

mysql>
mysql> source w3schools.sql;

# let's see whether the DB is there
mysql> use w3schools;
mysql> show tables;

mysql> SELECT customerName, city FROM customers;
\end{minted}

\bigskip
3. Go back to \emph|mysql-workbench| and choose w3schools DB as your default target for your queries.

\begin{minted}{sql}
-- let's explore
show tables;

-- go through all the tables
-- and display the top 5
select * from products limit 5

-- check schema:
describe table products
\end{minted}

\bigskip
3. Choose \verb|Database| $\rightarrow$ \verb|Reverse Engineer| to run a wizard, so we can learn more about the imported database.

%Zauważ rezultat będzie podobny do\\
%\href{https://www.w3schools.com/sql/trysql.asp?filename=trysql_select_columns}{https://www.w3schools.com/sql/trysql.asp?filename=trysql\_select\_columns} .

%%%%%%%%%%%%%%%%%%%%%%
\section{Exercises}

\subsection{SFW Queries}

\begin{enumerate}
 \item \verb|select|;
 \item \verb|select| and \verb|where|;
 \item \verb|select|, \verb|where|, and \verb|like|;
 \item \verb|select| and \verb|distinct|;
 \item \verb|select|, \verb|where|, and \verb|order by|;
 \item \verb|select|, \verb|where|, and \verb|count(*)|;
 \item \verb|update|
\end{enumerate}
% select distinct price from products order by price

\subsection{Multiple tables}

\begin{markdown}
- Supplier, productName
- ProductName, SupplierName, CategoryName
- OrderID, CustomerName, OrderDate
\end{markdown}

Notes:
\bigskip
\bigskip
\bigskip
\bigskip
\bigskip


% \mintinline{text}{add_SMOKE_TEST_to_README}, dodaj sekcję \mintinline{text}{## Smoke tests}
% \mintinline{bash}{git checkout} możemy anulować zmiany.
% \textit{x}
%
% \begin{minted}{bash}
% $ git tag -a v1.0.0 -m "Tests for SQL"
% #
% git push tag v1.0.0
% #
% $ git push --tags
% \end{minted}

% %%
% \pagebreak
% \bigskip

\end{document}
